\input{/Users/daniel/github/config/preamble.sty}%This is available at github.com/danimalabares/config
\input{/Users/daniel/github/config/thms-eng.sty}%This is available at github.com/danimalabares/config

%\usepackage[style=authortitle-terse,backend=bibtex]{biblatex}
%\addbibresource{/Users/daniel/github/config/bibliography.bib}

\newcommand{\rightlooparrow}{\mathbin{
    \vbox{\openup-10.25pt\halign{\hss$##$\hss\cr\circ\cr\longrightarrow\cr}}
}}

\begin{document}
\bibliographystyle{alpha}

\begin{minipage}{\textwidth}
	\begin{minipage}{1\textwidth}
		Complex geometry\hfill Daniel González Casanova Azuela
		
		{\small \hfill\href{https://github.com/danimalabares/cg}{github.com/danimalabares/cg}}
	\end{minipage}
\end{minipage}\vspace{.2cm}\hrule

\vspace{10pt}
{\huge Notes on complex geometry}
\tableofcontents


\section{Basic complex geometry}

Definition of \( \partial \) and \(\bar\partial\)…

\section{Sheaves}

\(\mathfrak{m}_x \subset C^\infty(\mathbb{R}^n)\) is the ideal of smooth functions vanishing at \(x \in \mathbb{R}^n\). In \cite{hart} definition of \textit{\textbf{local ring of \(P\) in \(Y\)}}, which is the ring of germs of regular functions of the variety \(Y\) at the point \(P\), that it is a local ring with maximal ideal \(\mathfrak{m}\), the set of germs of regular functions which vanish at \(P\). The residue field is \(k\).

\begin{exercise}\leavevmode
Show that indeed \(\mathfrak{m}_x\), the ideal of functions vanishing at \(x\), is maximal.
\end{exercise}
\begin{proof}[Solution]\leavevmode
	Suppose that \(\mathfrak{n} \supseteq \mathfrak{m}_x\) is another ideal contained in \(\mathcal{O}_{X,x}\). If \([f] \in \mathfrak{n}\) does not vanish at \(x\), then we can do \([1/f]\) very near  \(x\), giving \([f][1/f] \in \mathfrak{n}\), so that \(\mathfrak{n}=\mathcal{O}_{X,x}\). And if all \([f] \in \mathfrak{n}\) vanish at \(x\), we get \(\mathfrak{n}=\mathfrak{m}_x\).
\end{proof}

\section{Exponential exact sequence}

It's a short exact sequence of sheaves:
\[\begin{tikzcd}0\arrow[r]&\underline{\mathbb{Z}}\arrow[r]&\mathcal{O}(X)\arrow[r]&\mathcal{O}^*_X\arrow[r]&0\end{tikzcd}\]
where \(X\) is a complex manifold, \(\mathcal{O}(X)\) the sheaf of holomorphic functions and \(\mathcal{O} ^*(X)\) the sheaf of nowhere-vanishing holomorphic functions. Neither of these two sheaves have cohomologies that match either of \(H^{\bullet}(X,\mathbb{C})\) nor \(H^{\bullet}(X,\mathbb{R})\); however the constant sheaf \(\underline{\mathbb{Z}}\) does give the same cohomology as singular integer cohomology.

As exact sequences often do, this ones gives a cohomology long exact sequence
\[\begin{tikzcd}[column sep=small]
	\cdots\arrow[r]&H^{0}(X,\mathbb{Z})\arrow[r]&H^{0}(X,\mathcal{O}_X)\arrow[r]&H^{0}(X,\mathcal{O}_X^*)\arrow[r]&\leavevmode\\
	H^{1}(X,\mathbb{Z})\arrow[r]&H^{1}(X,\mathcal{O}_X)\arrow[r]&H^{1}(X,\mathcal{O}_X^*)\arrow[r]&\leavevmode\\
	\arrow[r,"c_1"]&H^{2}(X,\mathbb{Z})\arrow[r]&H^{2}(X,\mathcal{O}_X)\arrow[r]&\cdots
\end{tikzcd}\]
To see it probably you'd have to delve back into \v Cech, but \(H^{1}(X,\mathcal{O}_X^*)\) is the same as \(\operatorname{Pic}(X)\). So in the end a line bundle gets assigned the cohomology class of its curvature w.r.t the Chern connection, or what

And not only that but actually the intersection form is cup product of Chern classes

\section{Riemann-Roch}

\subsection{Genus}
\begin{defn}[\cite{hart}, p.180]\leavevmode
\(X\) nonsingular projective variety, \textit{\textbf{geometric genus}} of \(X\) is \(p_g:=\dim_k\Gamma(X,\omega_X)\), where \(\omega_X=\Lambda^{n}(\Omega_{X/k})\) is the canonical sheaf/bundle.
\end{defn}

\begin{remark}\leavevmode
I think these sections \(\Gamma(X,\omega_X)\) can also be written as \(H^{1}(X,\mathcal{O}_X)\). (To me it's not obvious why.)
\end{remark}

\begin{defn}[p. 54]\leavevmode
The hilbert polynomial \(P_Y\) of a projective variety  \(Y\) is the polynomial whose coefficients are the dimensions of every summand in the graded decomposition \(\bigoplus S^i\) of \(\mathcal{O}_X\) (using that \(Y\) is projective).

The \textit{\textbf{arithmetic genus}} of \(Y\) is \(p_a(Y):=(-1)^{r}(P_Y(0)-1).\)
\end{defn}

\begin{remark}\leavevmode
	In the case of a projective nonsingular \textit{curve}, the arithmetic genus and the geometric genus coincide (by Serre duality). This may not be true in dimension \(\geq 2\).
\end{remark}

\begin{thing4}{Proposition IV.1.1}\label{prop:IV.1.1}\leavevmode
If \(X\) is a curve, then
\[p_a(X)=p_g(X)=\dim_kH^{1}(X,\mathcal{O}_X),\]
so we call this number simply the \textit{\textbf{genus}} of \(\mathfrak{X}\) and denote it by \(g\).
\end{thing4}

\subsection{Euler characteristic}

\begin{thing4}{Definition}[\cite{hart}, p. 360]\label{def:}\leavevmode
For any coherent sheaf \(\mathcal{F}\),
\[\chi(\mathcal{F})=\sum(-1)^i\dim_kH^{i}(X,\mathcal{F}).\]
\end{thing4}

\begin{remark}\leavevmode
Once upon a time, after a long discussion in Donaldson I convinced myself that \(H^{0}(X,\mathcal{F})\) is the set of sections of the sheaf \(\mathcal{F}\).
\end{remark}

\subsection{For curves}
According to wikipedia, \(\ell(D)\) for a divisor \(D\) on a Riemann surface (\(D\) is a sum of points with some coefficients) is the set of all meromorphic functions \(h\) such that the coefficients of \((h)+D\) are non-negative.
\begin{thing4}{Theorem IV.1.3}[Riemann-Roch]\label{prop:IV.1.3}\leavevmode
Let \(D\) be a divisor on a curve \(X\) of genus \(g\). Them
\[\ell(D)-\ell(K-D)=\operatorname{deg}D-g+1.\]
\end{thing4}

\subsection{For surfaces}
\begin{thm}[K3 course]\leavevmode
\(L\) line bundle on a surface and \(K_X=\Omega^2(X)\) its canonical bundle. Then
\[\chi(L)=\chi(\mathcal{O}_X)+\frac{(L-K_X,L)}{2}.\]
\end{thm}

\section{Ampleness}

Here's the \textbf{upshot} about ampleness:

From II.7, subsection \textit{Ample Invertible Sheaves}, p. 153:
\begin{quotation}
	Now that we have seen that a morphism of a scheme \(X\) to a projective space can be characterized by giving an invertible sheaf on \(X\) and a suitable set of its global sections, […]

	Recall that in §5 we defined a sheaf \(\mathcal{L}\) on \(X\) to be \textit{\textbf{very ample relative to \(Y\)}} if there is an immersion \(i: X \to \mathbb{P}^n_Y\) for some \(n\) such that \(\mathcal{L} \cong i^*\mathcal{O}(1)\). In case \(Y = \operatorname{Spec}A\), this is the same thing as saying that \(\mathcal{L}\) admists a set of global sections \(s_0,\ldots,s_n\) such that the corresponding morphism \(X \to \mathbb{P}^n_A\) is an immersion.

	We have also seen (5.17) that if \(\mathcal{L}\) is a very ample invertible sheaf on a projective scheme \(X\) over a noetherian ring \(A\), then for any coherent sheaf \(\mathcal{F}\) on \(X\), there is an integer \(n_0>0\) such that for all \(n \geq  n-\), \(\mathcal{F} \otimes \mathcal{L}^n\) is generated by global sections. […]
	\begin{defn}\leavevmode
	An invertible sheaf \(\mathcal{L}\) on a noetherian scheme \(X\) is said to be \textit{\textbf{ample}} if for every coherent sheaf  \(\mathcal{F}\) on \(X\) there is an integer \(n_0>0\) (depending on \(\mathcal{F}\)) such that for every \(n \geq  n_0\) the sheaf \(\mathcal{F} \otimes \mathcal{L}^n\) is generated by its global sections. (Here \(\mathcal{L}^n:= \mathcal{L}^{\otimes n}\).)
	\end{defn}
	[…]
	\begin{thing5}{Remark II.7.4.3}\label{rk:II.7.4.3}\leavevmode
		[…] we will see below (7.6) that if \(\mathcal{L}\) is ample, then some tensor power  \(\mathcal{L}^m\) of \(\mathcal{L}\) is very ample.
	\end{thing5}
\end{quotation}

So, being very ample is just a term to hide the possibility of embedding the variety in a projective space and pulling back the hyperplane bundle to that line bundle. But it turns out that this is equivalent to being generated by global sections in some sense (maybe taking tensor product).

\begin{thing4}{Example II.6.4}\label{exer:}\leavevmode
	We will see later (IV, 3.3) that if  \(D\) is a divisor on a complete nonsingular curve \(X\), them \(\mathcal{L}(D)\) is ample iff \(\operatorname{deg}D>0\). This is a consequence of the Rimeann-Roch theorem.
\end{thing4}

\begin{question}\leavevmode
What's up with the base-points?
\end{question}

To answer we move along to subsection  \textit{Linear systems} in Hartshorne.
\begin{quotation}
	[…] global sections of an invertible sheaf correspond to effective divisors on a variety. Thus giving an invertibnle sheaf and a set of its global sections {\color{6}(which is related to finding the embedding of the variety into projective space pulling back the hyperplane bundle!)} is the same as giving a certain set of effective divisors, all linearly equivalent to each other.
\end{quotation}

\begin{thing6}{Idea}[dani]\leavevmode
That we can associate divisors to sections. Then we consider all linearly equivalent divisors to a given one (a linear system), which corresponds to a set of sections \(V \subseteq \Gamma(X,\mathcal{L})\).
\end{thing6}

\textbf{And then} 
\begin{thing4}{Lemma II.7.8}\label{lem:II.7.8}\leavevmode
	[…] In particular, [a linear system] \(\mathfrak{d}\) is base-point-free iff \(\mathcal{L}\) is generated by the global sections in \(V\).
\end{thing4}

\subsection{Line bundles that are the pullback of the hyperplane bundle}



\section{Adjunction formula}

This is really an application of Riemann-Roch for surfaces:

\begin{thing1}{Proposition V.1.5}[\cite{hart}]\leavevmode
If \(C\) is a nonsingular curve of genus \(g\) on the surface \(X\) and \(K\) is the canonical divisor on \(X\), then
\[2g-2=C.(C+K).\]

\end{thing1}

\subsection{Kodaira ampleness criterion}

In K3 lecture 9 we have

\begin{thm}[Kodaira]\leavevmode
A bundle \(L\) is very ample iff \(c_1(L)\) is a Kähler class.
\end{thm}

\begin{thing7}{Recall}\leavevmode
that a line bundle is \textit{\textbf{prequantizable}} if its curvature is symplectic and integral.
\end{thing7}

\section{Positive \((1,1)\)-forms}
Any positive \((1,1)\) form looks like this:  \(\sum \alpha I x_i \wedge x_i\) for some positive functions \(\alpha_i \geq 0\).

\section{Hodge Index Theorem}

\begin{thing4}{Theorem V.1.9}[Hodge Index Theorem]\label{prop:V.1.9}\leavevmode
Let
\end{thing4}


\section{Volume form}

{\color{6}dani:} It's
\[\text{constant} \prod \partial z_i \wedge \bar\partial z_i=\text{const.} \sum dx_i \wedge dy_i\]
{\color{6}voi:}It's
\[\frac{\omega^n}{n!}=\prod dz_i \wedge d\bar{z}_i\]
where \(dz_i=dx_i + \sqrt{-1}dy_i\). It is the volume form of the hermitian manifold, i.e. the unique nowhere-vanishing section of the determinant bundle that gives 1 to the volume of the real unit cube \(e_1 \wedge Ie_1\wedge\ldots\wedge e_n \wedge I e_n\) obtained from an \(h\)-orthonormal complex basis \(\{e_i\}\).


\section{Adjunction formula}

\section{Kähler metric}
The Kähler form is the differential of a plurisubharmonic function \(\psi\). that is \(\omega=d d^c \psi=\sqrt{-1} \partial \partial \psi\).

\section{Picard group, Neron-Severi group}

\begin{defn}[dani]\leavevmode
\textit{\textbf{Picard group}} is the group of line bundles with tensor product.
\end{defn}

\begin{remark}\leavevmode
See \cite{hart} p. 151 for the quick comment ``We have seen (6.17) that \(\operatorname{Pic}\mathbb{P}^n_k \cong\mathbb{Z}\) and is generated by \(\mathcal{O}(1)\)".
\end{remark}

\begin{thing4}{Definition}[\cite{huyk3}, p.5]\label{prop:1.2.1}\leavevmode
\textit{\textbf{Néron-Severy group}} of an algebraic surface \(X\) is the quotient
\[\operatorname{NS}(X):=\operatorname{Pic}(X) /\operatorname{Pic}_0(X)\]
by the connected component of the Picard variety \(\operatorname{Pic}(X)\), i.e. by the subgroup of line bundles that are \textit{algebraically} equivalent to zero (?).
\end{thing4}

\begin{thing4}{Definition}[\cite{huyk3}, p.5]\label{prop:}\leavevmode
\[\operatorname{Num}(X):=\operatorname{Pic}(X)/\operatorname{Pic}^\tau(X)\]
where \(\operatorname{Pic}^\tau(X)\) is the subgroup of \textit{numerically trivial} line bundles, i.e. line bundles \(L\) such that \((L.L')=0\) for all line bundles \(L'\). (E.g. any  \(L \in \operatorname{Pic}^0\) is numerically trivial.) 
\end{thing4}

\section{K3 surfaces}
\begin{defn}[dani]\leavevmode
A \textit{\textbf{K3}} surface is a complex surface with trivial canonical bundle and vanishing first (co)homology. (It's dimension 2 so 1st cohomology is first homology by Poincaré.)
\end{defn}

\begin{defn}[K3 course]\leavevmode
A \textit{\textbf{K3 surface}} is a complex surface \(M\) with \(b_1=0\) (Betti number is dimension of homology) and \(c_1(M,\mathbb{Z})=0\). Recall that the Chern class coming from the exponential sequence is \(\operatorname{Pic}(M) \xrightarrow{c_1}H^{2}(M,\mathbb{Z})\), so in particular it means that the first Chern class of the canonical bundle \(K_X \in \operatorname{Pic}(M)\) is trivial, which in turn makes it trivial via Hodge theory.
\end{defn}

\begin{remark}\leavevmode
\cite{huyk3} shows that K3 surfaces have (algebraic, what is that?) fundamental group in remark 2.3.
\end{remark}

\begin{defn}[\cite{huyk3}]\leavevmode
A \textit{\textbf{K3 surface}} over \(k\) is a complete non-singular variety \(X\) of dimension two such that
 \[\Omega^2_{X/k}\cong \mathcal{O}_X\qquad \text{and} \qquad H^{1}(X,\mathcal{O}_X)=0\]
\end{defn}

\begin{thing4}{Proposition 1.2.1}[\cite{huyk3}]\label{prop:1.2.1}\leavevmode
\(\operatorname{NS}(X)\) and its quotient  \(\operatorname{Num}(X)\) are finitely generated.

The rank of \(\operatorname{NS}(X)\) is called the \textit{\textbf{Picard number}} \(\rho(X):=\operatorname{rk}\operatorname{NS}(X)\).
\end{thing4}

\begin{thing4}{Proposition 1.2.4}[\cite{huyk3}]\label{prop:1.2.4}\leavevmode
For a K3 surface \(X\) the natural surjections are isomorphisms (\textbf{warning:} the second isomorphism might not hold for general complex K3 surfaces):
\[\operatorname{Pic}(X) \xrightarrow{\sim}\operatorname{NS}(X)\xrightarrow{\sim}\operatorname{Num}(X)\]
and the intersection pairing on \(\operatorname{Pic}(X)\) is even, non-degenerate, and of signature \((1,\rho(X)-1)\).
\end{thing4}



\section{Fubini-Study}
It's a metric, it's a symplectic form. \textit{\textbf{Fubini-Study (symplectic) form}} is a closed 2-form defined on \(\mathbb{C}P^{n}\) as the exterior differential of the logarithm of the length functions \(\ell=\sum |z_i|^2\), i.e. \(\omega=dd^c \operatorname{log} \ell\).

This also has a local expression in coordinates \((z_1,\ldots,z_n)\) that might be interesting.

The \textit{\textbf{Fubini-Study metric}} is \(g(\cdot ,\cdot )=\omega(\cdot ,I\cdot )\).

\section{Hypercomplex manifolds}

\begin{defn}\leavevmode
	A manifold $M$ is \textit{\textbf{hypercomplex}} if it has three integrable almost complex structures  $I$,  $J$, $K$ satisfying the quaternionic relations $I^2=J^2=K^2=-\operatorname{Id}$ and $I J=-J I=K$.
\end{defn}


\begin{remark}[Obata Connection, GPT]
    Given a hypercomplex manifold $(M, I, J, K)$, there exists a unique torsion-free connection $\nabla^{\operatorname{ob}}$ such that
    \[
    \nabla^{\operatorname{ob}} I = \nabla^{\operatorname{ob}} J = \nabla^{\operatorname{ob}} K = 0.
    \]
    This is called the \textit{Obata connection}. Unlike the Levi-Civita connection, it is not necessarily compatible with a metric. Instead, it preserves the entire hypercomplex structure and serves as the natural connection in hypercomplex geometry.
\end{remark}

\bibliography{bib.bib}

\end{document}
